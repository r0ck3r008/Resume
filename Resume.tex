\documentclass[9]{Resume}

\begin{document}

\renewcommand{\maketitle} {
    \begin{flushleft}
        \textbf{\Huge\theauthor}
    \end{flushleft}
    \begin{flushleft}
        \small{\textbf{{\faMapMarker} \hspace{0.2cm} Gainesville, Fl} \hspace{2.5cm} \textbf{{\faMobile} \hspace{0.2cm} +1 (352)--226--1217} \\
        \textbf{{\faEnvelope} \hspace{0.1cm} naman.arora@tutanota.com} \hspace{0.5cm}
        \textbf{{\faGithub} \hspace{0.1cm} \href{https://github.com/r0ck3r008}{https://github.com/r0ck3r008}}}
    \end{flushleft}
}

\title{Resume}
\author{Naman Arora}
\maketitle

\begin{minipage}[t]{0.45\textwidth}
    \vspace{-\baselineskip}
    \cvsect{{\faUser} About Me}

        I am a recent masters graduate with special focus on GNU/Linux and systems level programming.
        Being a passionate developer, I have keen interest in all things Databases, Networking, Operating Systems and Cybersecurity.
        From peer to peer overlay DHT networks to concurrent RDBMS, I have built systems with maintainable and extensible code-bases and look forward to working with people who love to take on biggest engineering challenges.
\end{minipage}
\hfill
\begin{minipage}[t]{0.45\textwidth}
    \vspace{-\baselineskip}
    \cvsect{{\faCogs} Skills}

    \begin{itemize}[noitemsep,nolistsep,leftmargin=*]
    \setlength\itemsep{-1em}
    \item[]\textit{Programming/Scripting}
            \vspace{-1em}
            \begin{multicols}{2}
            \begin{itemize}[leftmargin=*]
            \setlength\itemsep{-0.25em}
                \item[]\textbf{C}\hspace{48pt}\meter{5}
                \item[]\textbf{C++}\hspace{37pt}\meter{4}
                \item[]\textbf{Python}\hspace{25pt}\meter{3}
                \item[]\textbf{Go}\hspace{28pt}\meter{3}
                \item[]\textbf{Rust}\hspace{21.5pt}\meter{2}
            \end{itemize}
            \end{multicols}

    \item[]\textit{Popular Frameworks and Tools}
            \vspace{-1em}
            \begin{multicols}{2}
            \begin{itemize}[leftmargin=*]
            \setlength\itemsep{-0.25em}
                \item[]\textbf{Docker}\hspace{25pt}\meter{4}
                \item[]\textbf{Git}\hspace{42pt}\meter{4}
                \item[]\textbf{Ghidra}\hspace{11pt}\meter{3}
                \item[]\textbf{GDB}\hspace{21pt}\meter{5}
            \end{itemize}
            \end{multicols}

    \item[]\textit{Platforms}
            \vspace{-1em}
            \begin{multicols}{2}
            \begin{itemize}[leftmargin=*]
            \setlength\itemsep{-0.25em}
            \item[]\textbf{GNU/Linux}\hspace{5pt}\meter{5}
            \item[]\textbf{Android}\hspace{20pt}\meter{2}
            \item[]\textbf{Windows}\hspace{3pt}\meter{3}
            \end{itemize}
            \end{multicols}
    \end{itemize}
        
\end{minipage}

\cvsect{{\faGraduationCap} Education}

\begin{itemize}[noitemsep,nolistsep]
    \item[]\textbf{University Of Florida \textit{{\scriptsize Gainesville, USA}} \hfill \textit{August 2019 - May 2021}}
        \begin{itemize}[leftmargin=*]
            \setlength\itemsep{-0.25em}
            \item[\textbullet]{\footnotesize \textbf{Master of Science} in Computer and Information Sciences, \textbf{GPA 3.51/4.00}}
            \item[\textbullet]{\footnotesize \textbf{Relevent Coursework:}
                ``Database System Implementation'', ``Distributed Operating Systems'',\\
                ``Ethical Hacking and Penetration Testing'', ``Malware Reverse Engineering'',
                ``Blockchain: Optimization and Application''}
        \end{itemize}
    \item[]\textbf{SRM Institute Of Science and Technology \textit{{\scriptsize Chennai, India}} \hfill \textit{July 2015 - May 2019}}
        \begin{itemize}[leftmargin=*]
            \setlength\itemsep{-0.25em}
            \item[\textbullet]{\footnotesize \textbf{Bachelor of Technology} in Computer Science and Engineering, \textbf{CGPA 8.3/10.0}}
            \item[\textbullet]{\footnotesize \textbf{Relevant Coursework:}
                                        ``Network Programming'', ``Database Management Systems'', ``Operating Systems''}
        \end{itemize}
\end{itemize}

\cvsect{{\faNewspaperO} Publications}
\begin{itemize}[noitemsep,nolistsep]
    \item[]\textbf{A Relay for A SDN Controller \hfill \textit{February 2020}}
        \begin{itemize}[leftmargin=*]
            \setlength\itemsep{-0.25em}
            \item[\textbullet]Indian Patent Application No. 202041008146 dated March 06, 2020
            \item[\textbullet]Realtime communication between controllers within distributed SDN
        \end{itemize}
    \item[]\textbf{An Attendance System and Method Thereof \hfill \textit{December 2018}}
        \begin{itemize}[leftmargin=*]
            \setlength\itemsep{-0.25em}
            \item[\textbullet]Indian Patent Office Journal dated July 27, 2018, page No. 28230
            \item[\textbullet]System for attendance using facial recognition
        \end{itemize}
\end{itemize}

\cvsect{{\faCode} Projects}
\begin{itemize}[noitemsep,nolistsep]
    \item[]\textbf{Malware Analysis \hfill \textit{April 2021}}
        \begin{itemize}[leftmargin=*]
            \setlength\itemsep{-0.25em}
            \item[\textbullet]\textit{Ghidra, x64\_dbg, Python, Radare2} \hfill \href{https://github.com/r0ck3r008/malware-analysis}{{\scriptsize https://github.com/r0ck3r008/malware-analysis}}
            \item[\textbullet]Malware sample analysis reports with Ghidra/Rizin workspaces and memory dumps
			\item[\textbullet]Analyzed Families: \textbf{\textit{Carberp RAT, AveMaria RAT, Ryuk Ransomware and FritzFrog BotNet}}
        \end{itemize}

    \item[]\textbf{DFS: A Database Engine Written from Scratch \hfill \textit{April 2020}}
        \begin{itemize}[leftmargin=*]
            \setlength\itemsep{-0.25em}
            \item[\textbullet]\textit{Linux, C++, Google Test, flex/yacc} \hfill \href{https://github.com/r0ck3r008/database-from-scratch}{{\scriptsize https://github.com/r0ck3r008/database-from-scratch}}
            \item[\textbullet]Database engine tested with processing \textbf{1GB or 7,000,000+ records}
            \item[\textbullet]Complete with aggregate, join and select functionalities
        \end{itemize}

    \item[]\textbf{A Performance Benchmark of Image Encryption Algorithms \hfill \textit{December 2019}}
        \begin{itemize}[leftmargin=*]
            \setlength\itemsep{-0.25em}
            \item[\textbullet]\textit{C, Python, Image Encryption, RISC-V, ARM, x86} \hfill \href{https://github.com/r0ck3r008/arch-perf-benchmark}{{\scriptsize https://github.com/r0ck3r008/arch-perf-benchmark}}
            \item[\textbullet]Compares simulated \textbf{RISC-V, ARM and x86 ISA}
            \item[\textbullet]Image encryption benchmark framework for \textbf{RC4, Chirikov and Vigenere algorithms}
        \end{itemize}

    \item[]\textbf{Twitter Engine for Highly Concurrent Systems \hfill \textit{December 2019}}
        \begin{itemize}[leftmargin=*]
            \setlength\itemsep{-0.25em}
            \item[\textbullet]\textit{Elixir, Phoenix Framework, Websockets} \hfill \href{https://github.com/r0ck3r008/twitter-engine}{{\scriptsize https://github.com/r0ck3r008/twitter-engine}}
            \item[\textbullet]Distributed and Fault tolerant Twitter Engine \textbf{tested with 100,000 simultaneous user processes}
        \end{itemize}

    \item[]\textbf{Tapestry: A P2P Distributed Hash Table Network \hfill \textit{November 2019}}
        \begin{itemize}[leftmargin=*]
            \setlength\itemsep{-0.25em}
            \item[\textbullet]\textit{Elixir, Tapestry, Distributed Hash Tables} \hfill \href{https://github.com/r0ck3r008/tapestry\_p2p}{{\scriptsize https://github.com/r0ck3r008/tapestry\_p2p}}
            \item[\textbullet]Resilient P2P network capable of \textbf{connecting 2,000+ clients with reliable resource sharing}
            \item[\textbullet]Reliable resource sharing even \textbf{with 20\% failed nodes}
        \end{itemize}

\end{itemize}

\end{document}
